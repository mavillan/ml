\documentclass[spanish, fleqn]{article}
\usepackage[spanish]{babel}
\usepackage[utf8]{inputenc}
\usepackage{amsmath}
\usepackage{amsfonts,txfonts}
\usepackage{mathrsfs}
\usepackage[colorlinks, urlcolor=blue]{hyperref}
\usepackage{fourier}
\usepackage[top = 2.5cm, bottom = 2cm, left = 2cm, right = 2cm]{geometry}


\title{Tarea N°1 \\INF393: Máquinas de Aprendizaje}
\author{Martín Villanueva A.}
\date{3 de noviembre 2015}

\begin{document}
\maketitle

\thispagestyle{empty}

\section*{Introducción}

En es primera tarea, se tiene como objetivo la implementación y testeo de algoritmos para regresión lineal y regresión logística. En ambos casos la busqueda de  los mejores parámetros del modelo se realiza por medio de 	\textit{Gradiente Descendente} (batch y online) y \textit{Newton-Raphson}. Para la correcta selección de los \textit{hiperparámetros} se realiza 5-fold crossvalidation, intentando de este modo que los modelos resultantes no caigan en problemas de \textit{overfitting}.

\section*{Parte 1 - Regresión Lineal}

\begin{enumerate}
  \item
\end{enumerate}




\section*{Parte 2- Regresión Logística}

\begin{enumerate}
  \item
\end{enumerate}

\section*{Conclusiones}

\vfill\hfill MV/\LaTeXe
\end{document}
